% Draft manuscript for Experimental Mechanics (structure-oriented, template-agnostic).
\documentclass[11pt]{article}

\usepackage[margin=1in]{geometry}
\usepackage{graphicx}
\usepackage{booktabs}
\usepackage{amsmath}
\usepackage{amssymb}
\usepackage{siunitx}
\usepackage{hyperref}
\usepackage{microtype}

\title{Robust Stereo Auto-Calibration with Compact Direct Reconstruction Models\\
via Physics-Informed Digital Twins: Project Baseline and Data Specification}
\author{Author(s)\\
\small Affiliation(s)\\
\small Corresponding author: email@example.com}
\date{\today}

\begin{document}
\maketitle

\begin{abstract}
Stereo calibration in non-ideal optical systems (non-pinhole behavior, off-axis configurations, and spatially varying blur)
often becomes brittle when relying on explicit inverse geometry (e.g., pose estimation from perfect corner detections) and
hardware-specific assumptions. This work targets a robust, image-based stereo auto-calibration workflow trained on
physics-informed digital twins (OptiX ray tracing~\cite{optix}) and deployed as a compact direct reconstruction model mapping stereo
correspondences to 3D coordinates.

This manuscript reports the project specification and the current implemented baseline: a reproducible dataset format
including mandatory physical metadata (pixel pitch, crop/resize/binning), a CPU fallback synthetic renderer for planar
targets, virtual ChArUco target generation, configurable texture resampling, spatially varying blur approximations, and a
parametric geometric distortion model. These components establish an experimental foundation for subsequent OptiX-based
simulation, amortized calibration learning (CalibNet), and compact ray-field/mapping decoders. In addition, we document a
measurement-oriented strategy for ChArUco corner identification on degraded imagery, including a non-parametric
low-frequency plane ``ray-field'' baseline that relaxes the pinhole assumption.
\end{abstract}

\textbf{Keywords:} stereo calibration; digital twin; ChArUco; uncertainty; blur; geometric distortion; direct 3D mapping

\section{Introduction}
Experimental Mechanics applications frequently require accurate 3D measurements across a wide range of length scales,
from micro-scale systems (e.g., microscopy with complex optical interfaces) to meter-scale structures.
In such regimes, classical pinhole camera models and conventional stereo calibration workflows can degrade due to:
(i) non-pinhole optics and off-axis arrangements, (ii) spatially varying blur and contrast, (iii) sensor preprocessing
differences (binning, crop/region of interest (ROI), resizing), and (iv) algorithmic brittleness when corner extraction or pose estimation is imperfect.

The long-term goal of this project is a robust two-phase workflow:
\emph{(1) calibration} from a small set of stereo image pairs of a planar target (e.g., ChArUco) to a low-dimensional latent
vector $z$, and \emph{(2) production inference} using a compact direct model (ray-field or polynomial/Zernike-like mapping)
to reconstruct 3D from stereo correspondences without solving a fragile inverse problem online.

This draft focuses on what is already implemented to support a rigorous experimental pipeline: a stable dataset specification
and a CPU synthetic renderer baseline to enable early testing, ablations, and instrumentation before GPU ray tracing is integrated.

\section{Project overview and intended contributions}
The complete project targets the following contributions:
\begin{enumerate}
  \item A digital-twin-driven training workflow for stereo auto-calibration robust to non-pinhole effects and blur.
  \item A compact representation for production inference: either a decoded ray-field with analytical triangulation, or a direct
        compact mapping from $(u_L,v_L,u_R,v_R)$ to $(X,Y,Z)$.
  \item A dataset format and metadata conventions enabling portability across sensors and preprocessing pipelines.
  \item Quantitative validation protocols and out-of-distribution detection for confidence reporting.
\end{enumerate}

The remainder of this manuscript documents the \emph{current baseline implementation} supporting Items (2) and (3) and providing
the scaffolding required to evaluate Items (1) and (4) in later milestones.

\section{Dataset specification (v0)}
\subsection{Design principles}
The dataset format is designed to (i) remain stable when swapping the data generator (CPU fallback vs.\ OptiX), (ii) retain
enough metadata to reproduce physical units, and (iii) decouple sensor preprocessing from the learned latent parameters.
In particular, the pixel pitch is treated as mandatory to anchor blur and reconstruction in physical units.

\subsection{On-disk structure}
The recommended directory layout is:
\begin{verbatim}
dataset/v0/
  manifest.json
  train|val|test/
    scene_0000/
      meta.json
      frames.jsonl
      left/000000.png
      right/000000.png
      gt_points.npz
      gt_charuco_corners.npz   (optional, for ChArUco scenes)
\end{verbatim}

\subsection{Metadata (mandatory physical fields)}
Each scene contains a \texttt{meta.json} with:
\begin{itemize}
  \item Sensor/view metadata for left and right images: resolution, crop, resize factors, binning, \textbf{pixel pitch} (µm).
  \item Board metadata: target type (currently \texttt{charuco} or a grid fallback) and its physical dimensions (mm).
  \item Generator parameters (audit): camera model, baseline, blur/noise settings, and optional distortion coefficients.
\end{itemize}

\subsection{Ground-truth labels}
The baseline provides planar ground-truth correspondences:
\begin{itemize}
  \item \texttt{gt\_points.npz}: tuples \{frame\_id, $(X,Y,Z)$\textsubscript{world} [mm], $(u,v)$\textsubscript{left/right} [px]\}.
  \item \texttt{gt\_charuco\_corners.npz} (optional): deterministic inner-corner IDs and their corresponding GT projections.
\end{itemize}

\section{CPU fallback synthetic renderer}
\subsection{Rationale}
GPU ray tracing (OptiX) is the intended high-fidelity generator, but early-stage development benefits from a lightweight,
deterministic fallback that runs on any machine. The CPU baseline implements:
\begin{enumerate}
  \item per-pixel ray casting to a planar board,
  \item texture sampling of a virtual ChArUco board (or an analytic grid),
  \item physically parameterized blur using pixel pitch, including a spatially varying edge blur approximation,
  \item optional Brown-Conrady geometric distortion with consistent forward and inverse mappings.
\end{enumerate}

\subsection{Ray--plane rendering and coordinate conventions}
For each pixel center $(u,v)$ in the delivered image, a ray direction is generated in the camera frame.
The ray is intersected with the plane pose $(R,t)$ in world coordinates, producing plane-local coordinates $(x_p,y_p)$ in millimeters.
Intensity is then obtained by sampling the target texture at the corresponding coordinates.

Sensor preprocessing (crop/resize/binning) is treated explicitly in metadata and a deterministic pixel-to-sensor conversion.
This is critical to experimental portability: the same physical system may deliver images under different acquisition settings.

\subsection{Virtual ChArUco generation and texture interpolation}
The ChArUco board texture is generated using OpenCV's ArUco module~\cite{opencv,garrido2014aruco,romero2018charuco} and treated as a grayscale target texture.
Texture sampling can be performed with selectable interpolation (nearest, linear, cubic, Lanczos4), implemented via OpenCV
\texttt{remap} when available.

\subsection{Blur and noise models}
Blur is parameterized as a Gaussian point spread function (PSF) specified either in pixels or in micrometers (converted to pixels via pitch and preprocessing).
To emulate spatially varying blur (e.g., due to off-axis optics), a radial edge blur is approximated by blending two blurred images
(center vs.\ edges) using a smooth radial weight field.
Additive Gaussian noise is available as a minimal sensor-noise proxy; additional camera effects (Poisson noise,
photo-response non-uniformity (PRNU), dark signal non-uniformity (DSNU), saturation, gamma, quantization) are planned.

\subsection{Geometric distortion model}
To capture common lens-like geometric aberrations, the baseline supports Brown-Conrady distortion on normalized camera coordinates.
The forward model distorts $(x_n,y_n)=(X/Z,Y/Z)$ prior to converting to sensor coordinates, and the inverse model is applied iteratively
for pixel-to-ray generation~\cite{brown1966decentering}. Distortion coefficients are stored per view in the scene metadata.

\section{ChArUco corner identification: from detection to stable 2D measurements}
\subsection{Problem statement and motivation}
We consider a planar ChArUco board: a chessboard whose squares are augmented with ArUco fiducial markers~\cite{garrido2014aruco,romero2018charuco}.
The board geometry is known in its own coordinate system. For each acquired grayscale image, the
practical goal of this section is to obtain a set of \emph{stable} 2D measurements (in pixels) for ChArUco inner corners,
robust to blur, compression, and geometric aberrations, such that downstream calibration and stereo reconstruction are not
dominated by brittle corner localization.

Formally, let $\{(x_i,y_i)\leftrightarrow \mathbf{u}_i\}_{i=1}^N$ denote detected ArUco marker-corner correspondences between known board-plane
coordinates and measured pixel coordinates, and let $\{(x_j,y_j)\}_{j=1}^M$ denote the known board-plane coordinates of all ChArUco inner corners.
The identification task is to estimate $\hat{\mathbf{u}}_j\approx \mathbf{u}(x_j,y_j)$ for each ChArUco corner location, using the structure of
the planar mapping and the observed correspondences.

The key observation is that classical pipelines ``detect $\rightarrow$ interpolate corners $\rightarrow$ estimate pose'' are
highly sensitive to small 2D biases. We therefore treat ChArUco identification as an estimation problem with explicit
geometric structure (planar target) and evaluate competing strategies against deterministic 2D ground truth on synthetic
datasets.

\subsection{Notation and conventions}
\subsubsection{Pixel-coordinate convention}
The dataset uses a pixel-center convention where integer $(u,v)$ correspond to pixel centers (with a continuous mapping
implemented via $(u+0.5,v+0.5)$ in the pixel$\leftrightarrow$sensor conversion). Some OpenCV APIs effectively output corner
coordinates in a $(u+0.5,v+0.5)$ convention; the evaluation code applies the appropriate correction for fair comparison.

\subsubsection{Geometry and camera models}
\paragraph{Coordinates.}
We denote board-plane coordinates in millimeters by $(x,y)\in\mathbb{R}^2$ and image coordinates in pixels by
$(u,v)\in\mathbb{R}^2$. Homogeneous coordinates are written with a tilde:
\begin{equation}
\tilde{\mathbf{x}} = (x,y,1)^\top,\qquad \tilde{\mathbf{u}} = (u,v,1)^\top.
\end{equation}
Two non-zero homogeneous vectors are equivalent if they differ by a non-zero scalar, denoted $\tilde{\mathbf{u}} \sim
\tilde{\mathbf{u}}'$. We use the dehomogenization (projective division) operator $\pi$ to map a 3-vector to 2D image coordinates:
\begin{equation}
\pi\!\left((u',v',w')^\top\right) = \left(\frac{u'}{w'},\frac{v'}{w'}\right)^\top,\qquad w'\neq 0.
\end{equation}
In particular, for a homography $H$ we write $\mathbf{u}(x,y)=\pi(H\tilde{\mathbf{x}})$ to denote the corresponding inhomogeneous pixel
coordinates. To avoid ambiguity with the camera projection, $\pi(\cdot)$ is only this dehomogenization operator; the full distortion-aware
camera projection is denoted by $\Pi(\cdot)$ (defined below).

\paragraph{Parametric (central) camera model.}
For central pinhole models, a 3D point $\mathbf{P}=(X,Y,Z)^\top$ in the camera frame projects to normalized coordinates
$(x_n,y_n)=(X/Z,Y/Z)$. Under Brown--Conrady distortion~\cite{brown1966decentering}, distorted normalized coordinates
$(x_d,y_d)$ are
\begin{equation}
\begin{aligned}
r^2 &= x_n^2+y_n^2,\quad \alpha(r)=1+k_1 r^2 + k_2 r^4 + k_3 r^6,\\
x_d &= x_n\,\alpha(r) + 2p_1x_ny_n + p_2(r^2+2x_n^2),\\
y_d &= y_n\,\alpha(r) + p_1(r^2+2y_n^2) + 2p_2x_ny_n,
\end{aligned}
\end{equation}
and pixel coordinates are obtained via intrinsics $K$:
\begin{equation}
\begin{bmatrix}u\\v\\1\end{bmatrix}
=
\begin{bmatrix}
f_x & s & c_x\\
0 & f_y & c_y\\
0 & 0 & 1
\end{bmatrix}
\begin{bmatrix}x_d\\y_d\\1\end{bmatrix}.
\end{equation}

\paragraph{Planar homography.}
For a planar target, the most general projective mapping from the board plane to the image is a homography:
\begin{equation}
\tilde{\mathbf{u}} \sim H\,\tilde{\mathbf{x}},\qquad H\in\mathbb{R}^{3\times 3},\ \det(H)\neq 0,
\end{equation}
where $H$ is defined up to scale. For a pinhole camera observing a plane, $H$ admits a physical factorization
$H = K\,[\mathbf{r}_1\ \mathbf{r}_2\ \mathbf{t}]$ (for plane $Z=0$ in board coordinates)~\cite{hartley2003multiple}. However,
homography-based methods use $H$ purely as a plane mapping and do not require $K$.

\subsection{Identification methods}
We consider two families of identification methods:
\emph{(i) parametric} methods that require an explicit optical model (intrinsics and distortion), and
\emph{(ii) non-parametric} plane mappings that operate directly on board-to-image correspondences without requiring
$(f_x,f_y,c_x,c_y)$.

\subsubsection{OpenCV ChArUco baseline}
Markers are detected and subpixel-refined, then ChArUco inner corners are interpolated by the ChArUco routine. This is a
strong practical baseline but can remain limited on degraded imagery due to sensitivity in detection and interpolation.

\subsubsection{Homography second pass (non-parametric)}
Using detected ArUco marker corners, we estimate a global plane-to-image homography $H$ using Random Sample Consensus
(RANSAC)~\cite{fischler1981ransac} and project all ChArUco
inner corners by applying $\pi(H\tilde{\mathbf{x}})$.

\subsubsection{PnP second pass (parametric)}
When the optical system is adequately approximated by a pinhole camera with Brown distortion, we estimate the board pose
$(R,t)$ via a Perspective-$n$-Point (PnP) solver on marker corners and re-project all ChArUco corners using the estimated
pose and the known $(K,\mathbf{d})$. This can strongly reduce error under distortion but is inherently model-dependent.

\subsubsection{Plane ray-field second pass (non-parametric)}
To accommodate complex optics while retaining a low-frequency assumption, we model the plane-to-image mapping as
``projective base + smooth residual field'':
\begin{equation}
\mathbf{u}(x,y) = \pi(H\,\tilde{\mathbf{x}}) + \mathbf{r}(x,y),
\end{equation}
where $\mathbf{u}(x,y)=(u(x,y),v(x,y))^\top$. Given ArUco marker-corner correspondences
$\{(x_i,y_i)\leftrightarrow \mathbf{u}_i\}_{i=1}^N$ with $\mathbf{u}_i=(u_i,v_i)^\top$, the observed residual at correspondence $i$ is
\begin{equation}
\hat{\mathbf{r}}_i \;=\; \mathbf{u}_i - \pi(H\,\tilde{\mathbf{x}}_i)\in\mathbb{R}^2,
\end{equation}
expressed in pixels. The residual field $\mathbf{r}(x,y)$ provides a smooth interpolation/denoising of these residuals over
the board domain. This ``ray-field'' is restricted to the calibration plane (a 2D warp); extending it to a full per-pixel 3D
ray-field requires additional constraints (e.g., multiple board poses and/or multiple planes).

\paragraph{Why this reduces point uncertainty without an optical model.}
The goal of this second pass is to reduce the measurement uncertainty of ChArUco corner locations by exploiting only two
sources of structure: (i) the target is planar with known coordinates $(x,y)$, and (ii) the mapping from the target plane to
the image varies smoothly over the board for many real optical systems (a low-frequency assumption).

We model each detected ArUco corner as a noisy observation of an unknown smooth mapping:
\begin{equation}
\mathbf{u}_i \;=\; \mathbf{u}(x_i,y_i) + \boldsymbol\varepsilon_i,
\end{equation}
where $\boldsymbol\varepsilon_i$ captures detector noise, blur/compression bias, and occasional outliers. Rather than using
the raw measurements directly, we estimate a smooth residual field $\mathbf{r}(x,y)$ from $\{\hat{\mathbf{r}}_i\}$ by
regularized fitting (Section~5.4.4). Because $\mathbf{r}(x,y)$ is constrained to be low-frequency, the estimate at a query
point $(x_j,y_j)$ effectively aggregates information from many nearby correspondences, acting as a denoising filter on the
plane-to-image warp. This reduces variance in the predicted corner position
$\hat{\mathbf{u}}_j=\pi(H\tilde{\mathbf{x}}_j)+\mathbf{r}(x_j,y_j)$, at the cost of a bias--variance trade-off: very
high-frequency distortions cannot be represented and will be smoothed out. Robust loss functions further improve stability
by limiting the influence of outlier detections.

\subsubsection{K-field surrogate (Jacobian field, non-parametric)}
As an intermediate baseline, we consider a smoothed field of local first-order models (a Jacobian field) that approximates
the plane-to-image mapping locally. This surrogate is ``$K$-like'' in the sense that it maps metric displacements on the
target plane to pixel displacements, but it is not a literal per-marker pinhole intrinsic matrix.

\subsubsection{Assumptions summary}
Table~\ref{tab:charuco_assumptions} summarizes the key assumptions behind each identification method.
\begin{table}[t]
\centering
\small
\begin{tabular}{p{3.2cm} l c p{6.8cm}}
\toprule
Method & Family & Requires $(K,\mathbf{d})$ & Core assumption\\
\midrule
OpenCV ChArUco & non-parametric & no & local photometric refinement + interpolation is unbiased\\
Homography (2nd pass) & non-parametric & no & global plane-to-image mapping is well described by a homography\\
PnP (2nd pass) & parametric & yes & central pinhole projection + Brown distortion is adequate\\
Plane ray-field & non-parametric & no & homography + smooth residual captures low-frequency non-idealities\\
K-field surrogate & non-parametric & no & mapping is locally affine; Jacobian field varies smoothly\\
\bottomrule
\end{tabular}
\caption{Assumptions and dependencies of the ChArUco identification methods used in this work.}
\label{tab:charuco_assumptions}
\end{table}

\subsection{Estimation procedures}
\subsubsection{Homography estimation (RANSAC + normalized DLT)}
The homography $H$ is estimated from ArUco marker-corner correspondences between board coordinates and image measurements by
minimizing a reprojection error on inliers:
\begin{equation}
\min_{H}\sum_{i\in\mathcal{I}} \left\| \mathbf{u}_i - \pi\!\left(H\,\tilde{\mathbf{x}}_i\right)\right\|_2^2,
\end{equation}
where $\mathcal{I}$ is the inlier set found by RANSAC~\cite{fischler1981ransac}.

\noindent
RANSAC repeatedly samples minimal correspondence sets, fits a candidate $H$, and selects inliers under a pixel reprojection
threshold. Given inliers, we compute $H$ using the Direct Linear Transform (DLT) algorithm~\cite{hartley2003multiple}.
DLT starts from the homogeneous constraint $\tilde{\mathbf{u}}_i \times (H\,\tilde{\mathbf{x}}_i)=\mathbf{0}$, which yields two
independent linear equations in the nine entries of $H$ per correspondence. Stacking all inlier equations produces a linear
system $A\,\mathbf{h}=\mathbf{0}$, with $\mathbf{h}$ the vectorized homography. The solution is obtained as the right singular
vector of $A$ associated with the smallest singular value and reshaped into a $3\times 3$ matrix. To improve numerical
conditioning, the normalized DLT variant is used: both $\tilde{\mathbf{x}}_i$ and $\tilde{\mathbf{u}}_i$ are pre-normalized by
similarity transforms such that their centroids are at the origin and their mean distance to the origin is $\sqrt{2}$, then
the resulting homography is de-normalized~\cite{hartley2003multiple}.

\subsubsection{Pixel-to-ray conversion (pinhole + undistortion)}
For parametric methods, we rely on the \emph{central projection} (pinhole) assumption: all viewing rays pass through a single
camera center (the origin of the camera frame). In this model, normalized coordinates are not merely a change of units: they
parameterize ray directions.

\paragraph{Normalized coordinates as ray directions.}
Consider the normalized image plane $Z=1$ in the camera frame. A 3D point $\mathbf{P}^c=(X,Y,Z)^\top$ with $Z>0$ defines
normalized coordinates $(x_n,y_n)=(X/Z,Y/Z)$, which are exactly the intersection of the ray $\lambda\,\mathbf{P}^c$ with the
plane $Z=1$:
\begin{equation}
(x_n,y_n,1)^\top \;=\; \left(\frac{X}{Z},\frac{Y}{Z},1\right)^\top.
\end{equation}
Conversely, a normalized coordinate pair $(x_n,y_n)$ defines the set of 3D points projecting to the same image location:
\begin{equation}
\mathbf{P}^c(\lambda) \;=\; \lambda\,(x_n,y_n,1)^\top,\qquad \lambda>0,
\end{equation}
which is a ray through the camera center. Therefore, the unit ray direction associated with $(x_n,y_n)$ is
\begin{equation}
\mathbf{d} \;=\; \frac{(x_n,y_n,1)^\top}{\left\|(x_n,y_n,1)^\top\right\|_2},
\end{equation}
which defines a central ray through the camera center~\cite{hartley2003multiple}.

\paragraph{From pixels to normalized coordinates.}
Let $\tilde{\mathbf{u}}=(u,v,1)^\top$ denote a pixel (homogeneous) coordinate. Without distortion, normalized coordinates satisfy
$\tilde{\mathbf{u}}\sim K(x_n,y_n,1)^\top$, hence $(x_n,y_n)^\top=\pi(K^{-1}\tilde{\mathbf{u}})$.
With Brown--Conrady distortion, we first compute \emph{distorted} normalized coordinates
\begin{equation}
(x_d,y_d)^\top=\pi\!\left(K^{-1}\tilde{\mathbf{u}}\right),
\end{equation}
then iteratively invert the distortion mapping to recover $(x_n,y_n)$~\cite{brown1966decentering}.

\noindent
In our metadata-aware formulation, $(f_x,f_y,c_x,c_y)$ are derived from the focal length in micrometers $f_{\mu m}$, the pixel pitch
and preprocessing parameters (crop/resize/binning), ensuring that blur expressed in micrometers can be related consistently to pixels.
This is analogous in spirit to calibration models that explicitly account for sensor geometry and preprocessing~\cite{zhang2000flexible}.

\noindent
\textbf{Remark (non-central optics).} The above ray construction relies on central projection. In non-central systems (e.g., some
microscope/CMO configurations), a pixel may not correspond to a ray passing through a single shared center. This motivates the
non-parametric plane mapping baselines in Section~5.3, which avoid imposing a pinhole ray model when it is not appropriate.

\noindent
\textbf{Important:} in the current synthetic baseline, $f_{\mu m}$ and distortion coefficients are known because they are part of the
scene metadata emitted by the generator. This makes the PnP-based method a controlled upper bound for the effect of corner
identification. For real data, these parameters must be estimated (e.g., via multi-view calibration) or replaced by a
non-parametric model such as the proposed plane ray-field.

\subsubsection{PnP pose estimation}
Let $\mathbf{P}_i=(x_i,y_i,0)^\top$ denote the 3D coordinates (in millimeters) of detected ArUco marker corners in the board coordinate system,
and let $\mathbf{u}_i$ be their measured pixel coordinates. Given known $(K,\mathbf{d})$, PnP estimates the board pose $(R,t)$ by minimizing
the reprojection error
\begin{equation}
\min_{R,t}\sum_{i\in\mathcal{I}} \left\| \mathbf{u}_i - \Pi\!\left(K,\mathbf{d},R,t,\mathbf{P}_i\right)\right\|_2^2,
\end{equation}
where $\mathcal{I}$ is an inlier set found by a robust scheme (e.g., PnP-RANSAC as implemented in OpenCV~\cite{opencv}). The projection
operator is defined by: $\mathbf{P}_i^c=R\mathbf{P}_i+t$, $(x_n,y_n)=(X_i^c/Z_i^c,Y_i^c/Z_i^c)$, then $(x_d,y_d)$ is obtained by Brown
distortion, and finally
\begin{equation}
\Pi(K,\mathbf{d},R,t,\mathbf{P}_i)=\pi\!\left(K\,(x_d,y_d,1)^\top\right).
\end{equation}
ChArUco inner corners are then predicted by projecting their known 3D board coordinates using the estimated pose.

\subsubsection{Plane ray-field estimation (objective and IRLS)}
The residual field $\mathbf{r}(x,y)$ is parameterized by node values on a regular grid in board coordinates, and evaluated
by bilinear interpolation. Let $\mathbf{r}_\theta(x_i,y_i)$ denote the interpolated residual at correspondence $i$, given
parameters $\theta$ (the stacked node values). We estimate $\theta$ by minimizing a robust, regularized objective:
\begin{equation}
\min_{\theta}\sum_i \rho\!\left(\left\|\mathbf{r}_\theta(x_i,y_i) - \hat{\mathbf{r}}_i\right\|_2^2\right)
 + \lambda \left\|L\,\theta\right\|_2^2,
\end{equation}
where $L$ is a discrete Laplacian operator on the grid. We use a Huber loss on the squared residual norm $t$:
\begin{equation}
\rho(t)=
\begin{cases}
t, & \sqrt{t}\le \delta,\\
2\delta \sqrt{t} - \delta^2, & \sqrt{t}>\delta,
\end{cases}
\end{equation}
which behaves quadratically near the origin and linearly for large residuals~\cite{huber1964robust}. Here, $\delta>0$ is a
user-chosen threshold in pixels. In practice, we solve this objective using iteratively reweighted least squares (IRLS),
with weights derived from $\rho'(t)$. Writing $\mathbf{s}_i(\theta)=\mathbf{r}_\theta(x_i,y_i)-\hat{\mathbf{r}}_i$ and
$s_i(\theta)=\|\mathbf{s}_i(\theta)\|_2$, the Huber weights for the squared residual norm take the simple form
$w_i=1$ if $s_i\le \delta$ and $w_i=\delta/s_i$ otherwise.

\noindent
The regularization term $\|L\theta\|_2^2$ uses a 4-neighborhood discrete Laplacian on the grid, enforcing smoothness consistent with a
low-frequency assumption.

\subsubsection{K-field estimation (local linearization and smoothing)}
To bridge between a single global intrinsic matrix $K$ and a fully non-parametric warp, we also consider a ``$K$-field''
surrogate based on a smoothed local \emph{first-order} approximation of the plane-to-image mapping. The key idea is that,
over a sufficiently small neighborhood on the calibration plane, many optical mappings are well approximated by an affine
model even when the global mapping is not.

Importantly, a \emph{true} pinhole intrinsic matrix $K$ is a global camera parameter. From a single planar view, only a
plane homography is identifiable, and recovering $K$ requires multiple views and additional constraints (e.g., Zhang's
calibration method~\cite{zhang2000flexible}). Therefore, the ``$K$-field'' considered here should be understood as a
measurement-driven local linearization (a Jacobian field) rather than a literal per-marker pinhole $K$.

Let $(x,y)$ denote board-plane coordinates and $(u,v)$ image coordinates. For a query location $(x_q,y_q)$ (in practice, the
center of an ArUco marker or a grid anchor), we fit a local affine mapping
\emph{equivalently obtained from a first-order Taylor expansion} of an unknown smooth mapping:
\begin{equation}
\begin{aligned}
u(x,y) &\approx u(x_q,y_q) + \left.\frac{\partial u}{\partial x}\right|_q (x-x_q) + \left.\frac{\partial u}{\partial y}\right|_q (y-y_q),\\
v(x,y) &\approx v(x_q,y_q) + \left.\frac{\partial v}{\partial x}\right|_q (x-x_q) + \left.\frac{\partial v}{\partial y}\right|_q (y-y_q),
\end{aligned}
\end{equation}
where the $2\times 2$ Jacobian
\begin{equation}
J(x_q,y_q)=
\begin{bmatrix}
\left.\frac{\partial u}{\partial x}\right|_q & \left.\frac{\partial u}{\partial y}\right|_q\\
\left.\frac{\partial v}{\partial x}\right|_q & \left.\frac{\partial v}{\partial y}\right|_q
\end{bmatrix}
\end{equation}
captures the local conversion from metric displacements on the calibration plane to pixel displacements.

\begin{equation}
\begin{aligned}
u(x,y) &\approx a_0(x_q,y_q) + a_1(x_q,y_q)\,x + a_2(x_q,y_q)\,y,\\
v(x,y) &\approx b_0(x_q,y_q) + b_1(x_q,y_q)\,x + b_2(x_q,y_q)\,y,
\end{aligned}
\end{equation}
using weighted least squares on nearby ArUco marker-corner correspondences $\{(x_i,y_i)\leftrightarrow(u_i,v_i)\}$:
\begin{equation}
\min_{\mathbf{a}} \sum_i w_i(x_q,y_q)\,\left(u_i - [x_i\ y_i\ 1]\mathbf{a}\right)^2,\quad
\min_{\mathbf{b}} \sum_i w_i(x_q,y_q)\,\left(v_i - [x_i\ y_i\ 1]\mathbf{b}\right)^2,
\end{equation}
with Gaussian weights $w_i=\exp\!\left(-\frac{\|(x_i,y_i)-(x_q,y_q)\|^2}{2\sigma^2}\right)$.

The local Jacobian of this affine approximation,
\begin{equation}
J(x_q,y_q)=
\begin{bmatrix}
a_1(x_q,y_q) & a_2(x_q,y_q)\\
b_1(x_q,y_q) & b_2(x_q,y_q)
\end{bmatrix},
\end{equation}
can be interpreted as an estimate of the Jacobian above. In that sense, the set of parameters $(a_0,a_1,a_2,b_0,b_1,b_2)$
is ``$K$-like'': it plays the role of locally mapping metric displacements on the target plane to pixel displacements,
without requiring a globally valid pinhole parameterization.

To enforce the low-frequency assumption, we estimate these parameters on a coarse grid of anchors in board coordinates and
then smooth each parameter field independently (Gaussian smoothing). For an arbitrary point $(x,y)$, the local parameters
are recovered by bilinear interpolation on the grid and the affine mapping is applied. While this surrogate can be useful
as an intermediate baseline, it is fundamentally limited by the expressiveness of an affine local model; in our experiments,
a projective base plus smooth residual field (the plane ray-field in the previous subsection) is typically more accurate.

\subsection{Evaluation protocol and metrics}
On synthetic datasets, ground-truth ChArUco corners are provided as stable IDs with their corresponding 2D image coordinates.
For each view (left/right), we match predictions to ground truth by ChArUco corner ID and compute a per-corner Euclidean error
$e_i=\|\hat{\mathbf{u}}_i-\mathbf{u}_i^{gt}\|_2$ in pixels. We report the root-mean-square (RMS) and 95th percentile (P95)
of $\{e_i\}$ for each view.

\subsection{Preliminary quantitative comparison}
Table~\ref{tab:charuco_methods} reports a baseline comparison between direct OpenCV ChArUco, homography and PnP second
passes, and the proposed plane ray-field baseline. The table is generated reproducibly by
\texttt{paper/experiments/compare\_charuco\_methods.py} and included here via \texttt{\\input}.

\input{tables/charuco_methods.tex}

\paragraph{Reading of the results.}
On clean synthetic imagery, planar geometric second passes can approach sub-tenth-pixel accuracy, indicating that most of
the error of the direct ChArUco pipeline is not purely numerical but induced by detection/interpolation sensitivity.
With blur and geometric distortions, PnP can outperform a global homography when the camera model is correct, whereas the
plane ray-field provides a competitive non-parametric alternative when pinhole assumptions do not hold.

\section{Preliminary checks and reproducibility}
The repository includes a dataset validator and an ``oracle'' evaluator that reprojects GT 3D points using the same model parameters
recorded in metadata. This ensures the data pipeline is internally consistent (e.g., units, coordinate conventions, preprocessing).

\section{Planned next steps (toward a full Experimental Mechanics submission)}
Immediate next steps toward a complete paper include:
\begin{itemize}
  \item OptiX-based digital twin with physically meaningful illumination/material effects and richer optical interfaces~\cite{optix}.
  \item Amortized calibration learning (CalibNet) with set-based encoders and uncertainty outputs.
  \item Compact decoder benchmarks: ray-field vs.\ direct mapping, speed/accuracy trade-offs, conditioning and confidence metrics.
  \item Extension from plane-only non-parametric warps to a full 3D per-pixel ray-field (and stereo triangulation) using multiple
        target poses and/or multiple calibration planes.
  \item Real-world validation protocol and sim-to-real ablations.
\end{itemize}

\section*{Data and code availability}
This manuscript is accompanied by a reproducible codebase and dataset format. The current implementation includes a CPU data generator,
dataset validator, and oracle evaluator. Packaging and public release details will be finalized in the full submission.

\section*{Implementation summary (current baseline)}
For traceability, the current baseline components correspond to:
\begin{itemize}
  \item Metadata schema and validation (pixel pitch mandatory) and preprocessing-aware pixel$\leftrightarrow$sensor conversions.
  \item CPU renderer with ray--plane intersection, virtual ChArUco texture generation, and selectable texture interpolation.
  \item Blur model with optional edge-varying approximation; additive noise.
  \item Optional Brown-Conrady distortion with forward projection and inverse pixel-to-ray mapping.
  \item Dataset validator and oracle evaluator for internal consistency checks.
  \item ChArUco identification evaluation vs.\ GT, including multi-stage geometric second passes and a plane-only non-parametric
        low-frequency ``ray-field'' baseline.
\end{itemize}

\section*{Limitations of the current baseline}
The CPU renderer is intentionally simplified: single planar target, direct texture sampling, and no physically based illumination.
OptiX-based rendering with materials, lighting, and optical interfaces is planned for future milestones.

\bibliographystyle{plain}
\bibliography{references}

\end{document}
